%Szablon przygotowany przez mgr Marcina Hanca, z drobnymi zmianami dr Michała Rena

\documentclass[12pt,a4paper,leqno,oneside,titlepage]{book}

% Wczytanie pakietów: kodowania, czcionki i języki.
\usepackage[utf8]{inputenc}
\usepackage{lmodern}
\usepackage[english,polish]{babel}
% Wczytanie pakietu 'polski' w celu zapewnienia polskich nazw.
\usepackage{polski}
% Czcionki matematyczne.
\usepackage{amsfonts}
\usepackage{amsmath}

% Ładne początki rozdziałów (pakiet fncychap).
% Polecam Sonny i Conny. Bjornstrup najładniejszy, ale mi się bugował.
\usepackage[Sonny]{fncychap}
% Ładne i klikalne odnośniki.
\usepackage{url}
% Odnośniki dla adresów z polskimi znakami.
\usepackage[]{hyperref}
% Możliwość tworzenia łączonych pól (wg. rzędów) w tabelach.
\usepackage{multirow}
% Pakiet do cytowania kodów źródłowych.
\usepackage{listings}
% Pakiet do ładnego wstawiania grafik.
\usepackage{graphicx}
% Pakiet dodający możliwość wstawienia rozdziału "Akronimy".
\usepackage{acronym}
% Pakiet dodający kolory
\usepackage[usenames,dvipsnames,svgnames,table]{xcolor}
% Pakiet rozwiązujący problem z underscore w Section.
\usepackage[T1]{fontenc}
% Pakiet dodający definicje i twierdzenia.
\usepackage{amsthm}

\frenchspacing

\author{Dawid Drozd}
\title{Dekompilacja aplikacji działających w systemie Android OS.}

% \imod{k} Ładny zapis dzielenia modulo.
\makeatletter
\def\imod#1{\allowbreak\mkern10mu({\operator@font mod}\,\,#1)}
\makeatother

% \rom{n} Liczba n zapisana rzymsko.
\makeatletter
\newcommand*{\rom}[1]{\expandafter\@slowromancap\romannumeral #1@}
\makeatother

% Własne definicje.
% \begin{mydef}
%     Treść definicji.
% \end{mydef}
\newtheorem{mydef}{Definicja}

% Ładny sposób wstawiania cytatu rozpoczynającego rozdział.
% \begin{chapquote}{KTO}
%     CO ONA POWIEDZIAŁA?
% \end{chapquote}
\makeatletter
\renewcommand{\@chapapp}{}
\newenvironment{chapquote}[2][2em]
  {\setlength{\@tempdima}{#1}%
   \def\chapquote@author{#2}%
   \parshape 1 \@tempdima \dimexpr\textwidth-2\@tempdima\relax%
   \itshape}
  {\par\normalfont\hfill--\ \chapquote@author\hspace*{\@tempdima}\par\bigskip}
\makeatother

% Zmiana tekstów w listingach kodów źródłowych na j. polski.
\renewcommand\lstlistingname{Kod źródłowy}
\renewcommand\lstlistlistingname{Spis kodów źródłowych}

% Redefinicja Abstract'ów.
% W celu możliwości wstawienia dwóch na jedną stronę.
\newenvironment{abstractpage}
  {\cleardoublepage\vspace*{\fill}\thispagestyle{empty}}
  {\vfill\cleardoublepage}
\newenvironment{abstract}[1]
  {\bigskip\selectlanguage{#1}%
   \begin{center}\bfseries\abstractname\end{center}}
  {\par\bigskip}

% Dodatkowe definicje stylu stron.
\lstset{
  basicstyle={\small\ttfamily},
  breaklines=true,
  columns=flexible
}

\setlength{\oddsidemargin}{0.5in}
\setlength{\textwidth}{5.7in}
\setlength{\topmargin}{0in}
\setlength{\textheight}{8.5in}
\linespread{1.05}

% Tu rozpoczyna się zawartość pracy!
\begin{document}

% Strona tytułowa zgodna z wymaganiami:
% http://www.wmi.amu.edu.pl/pl/prace-dyplomowe
\begin{titlepage}
\let\footnotesize\small
\let\footnoterule\relax
\let \footnote \thanks

\begin{center}
{\large \bf Uniwersytet im. Adama Mickiewicza w Poznaniu \\ Wydział Matematyki i~Informatyki \par}
\vspace{0.5cm plus 1mm minus 2mm}
{{\bf Kierunek: Informatyka} \\ \small Specjalizacja:\par}
\end{center}%

\vspace{1.5cm plus 1fill}
\begin{flushleft}
{\center {\bf \Large Dawid Patryk Drozd} \\ \normalsize Nr albumu: \bf 362617\par}
\end{flushleft}
\vspace{1.5cm plus 1mm minus 2mm}

\begin{center}
{\huge\textbf{Dekompilacja aplikacji działających w systemie Android~OS}\par}
\vspace{0.5cm plus 1mm minus 2mm}
{\large Decompiling Android OS applications}
\par
\vspace{1.5cm plus 1.5fill}

\begin{flushright}\large
\begin{tabular}{l}
Praca magisterska\\[3pt]
\MakeUppercase{ }\\[3pt]
Promotor: \\[3pt]
\bfseries dr Michał Ren \\[3pt]
\end{tabular}
\end{flushright}
\vspace{4cm plus .1fill}
{\large 2017\par}
\end{center}
\end{titlepage}

% Zgłoszenie braku numerowania kolejnych stron.
\pagenumbering{gobble}

\begin{flushright}{
Poznań, dnia 25 czerwca 2017
}\end{flushright}
\begin{center}{
\par
\vspace{1.5cm plus 1.5fill}
{\large OŚWIADCZENIE}
}\end{center}
\par
\vspace{1.5cm plus 1.5fill}
Ja, niżej podpisany Dawid Patryk Drozd student Wydziału Matematyki i~Informatyki Uniwersytetu im. Adama Mickiewicza w Poznaniu oświadczam, że przedkładaną pracę dyplomową pt: ``Dekompilacja aplikacji działających w systemie Android OS'' napisałem samodzielnie. Oznacza to, że przy pisaniu pracy, poza niezbędnymi konsultacjami, nie korzystałem z pomocy innych osób, a~w~szczególności nie zlecałem opracowania rozprawy lub jej części innym osobom, ani nie odpisywałem tej rozprawy lub jej części od innych osób.\\

Oświadczam również, że egzemplarz pracy dyplomowej w~wersji drukowanej jest całkowicie zgodny z~egzemplarzem pracy dyplomowej w~wersji elektronicznej.\\

Jednocześnie przyjmuję do wiadomości, że przypisanie sobie, w~pracy dyplomowej, autorstwa istotnego fragmentu lub innych elementów cudzego utworu lub ustalenia naukowego stanowi podstawę  stwierdzenia  nieważności postępowania w~sprawie nadania tytułu zawodowego.\\

Wyrażam zgodę na udostępnianie mojej pracy w czytelni Archiwum UAM.\\

Wyrażam zgodę na udostępnianie mojej pracy w zakresie koniecznym do ochrony mojego prawa do autorstwa lub praw osób trzecich.
\par
\vspace{1.5cm plus 1.5fill}
\begin{center}{
..............................................\\
{\footnotesize(czytelny podpis studenta)}
}\end{center}

\newpage

\phantom{.}

\vspace{12cm} \hspace{1cm}\phantom{.}\\
\phantom{.}\hspace{5cm}{Składam serdeczne podziękowania}\\
\phantom{.}\hspace{5cm}{doktorowi }\\
\phantom{.}\hspace{5cm}{Michałowi Renowi }\\
\phantom{.}\hspace{5cm}{za jego nieocenioną pomoc }\\
\phantom{.}\hspace{5cm}{przy pisaniu tej pracy.}\\
\phantom{.}\hspace{5cm}{}\\
\phantom{.}\hspace{5cm}{Dziękuję również}\\
\phantom{.}\hspace{5cm}{Marcinowi Hancowi za szablon i }\\
\phantom{.}\hspace{5cm}{za owocną, naukową współpracę}\\
\phantom{.}\hspace{5cm}{przy tworzeniu wyników z tej pracy.}\\

\newpage
% Przód pracy - spisy i abstrakty.
\frontmatter
% Spis treści ze specjalnym uwzględnieniem podkreśleń w tytułach sekcji.
\pagestyle{plain}
{
    \catcode`\_=12
    \tableofcontents
}
% Spis ilustracji.
\listoffigures
% Spis tabeli.
\listoftables
% Spis listingów kodów źródłowych.
\begingroup
\let\clearpage\relax
\lstlistoflistings
\endgroup

% Strona z abstraktami.
\begin{abstractpage}
% Abstrakt w języku polskim.
\begin{abstract}{polish}
Streszczenie wstępu jest dobrym pomysłem na abstrakt. Dobre praktyki tworzenia abstraktów znajdują się na stronie\footnote{
\url{http://www.editage.com/insights/how-to-write-an-effective-title-and-abstract-and-choose-appropriate-keywords}}.
\end{abstract}
\smallskip
\noindent \textbf{Słowa~kluczowe:} schemat pracy dyplomowej

%Abstrakt w języku angielskim.
\begin{abstract}{english}
Translation of your polish abstract.
\end{abstract}
\smallskip
\noindent \textbf{Keywords:} thesis schema
\end{abstractpage}

% Finally - PRACA!
\mainmatter

% Wstęp jest uwzględniony w spisie treści jako rozdział bez numeru.
\addcontentsline{toc}{chapter}{Wstęp}
\chapter*{Wstęp}

Krótkie omówienie -- skąd pomysł na pracę, być może kilka słów wprowadzenia w historię badań nad podobnymi tematami.\\

Wstęp powinien być przewodnikiem po własnych wynikach. Czyli w~pracy takiej zrobiono takie owakie cuda. W rozdziale tym to, a w tamtym tamto.

\chapter{Typowy rozdział}

Blablabla.

\section{O! Podrozdział!}

% Wstawianie ilustracji o określonej szerokości.
\begin{figure}[h!]
  % Centrowanie obrazu w poziomie.
  \centering
    \includegraphics[width=0.35\textwidth]{logo_wersja-podstawowa_granat_1.jpg}
  % Warto podawać źródło - np. z pomocą cite w podpisie.
  \caption{TRNG\cite{Stallings11kryptografia}}
\end{figure}

Dobrze jest cytować\footnote{Inaczej przytaczać. A to jest przypis dolny.} artykuły naukowe w taki sposób, aby czytający nie musiał co chwilę szukać bibliografii, np. książka ``\textit{Kryptografia i bezpieczeństwo sieci komputerowych. Matematyka szyfrów i techniki kryptologii}''\cite{Stallings11kryptografia}, strona internetowa ``\textit{Seminarium ZATABEDA}''\cite{Gogolewski13seminarium}, artykuł zamieszczony w Internecie ``\textit{How to write an effective title and abstract and choose appropriate keywords}''\cite{Rodrigues13howtowrite}, wiadomość z grup dyskusyjnych ``\textit{Random numbers for C: The END?}''\cite{Marsaglia99randomnum} i prezentacja lub wykład zamieszczone w Internecie ``\textit{Szumy pseudolosowych map.}''\cite{Hanc15szumy}.\\

``Brzydki sposób na wstawianie cytatu.''.

% Lista nienumerowana.
\begin{itemize}
\item \textbf{Wymieniany} -- element może mieć zagnieżdżenia:
    \begin{itemize}
    \item \textbf{Przykład}, faktycznie tu jest.
    \end{itemize}
\end{itemize}

% Własna definicja
\begin{mydef}
\textbf{Można pogrubić nazwę określanego pojęcia} aby było jasne co definiujemy.
\end{mydef}

\chapter{Inny rozdział}

% Ilustracje mogą zawierać naraz kilka plików.
% Pamiętaj o podawaniu źródła wykorzystywanych obrazów.
% Nawet jeśli stworzyłaś/-eś reprodukcję.
\begin{figure}[h!]
  \centering
    \includegraphics[height=0.3\textheight]{PM5544_with_non-PAL_signals.png}
    \includegraphics[height=0.3\textheight]{logo_wersja-uzupeniajca_czarny_2.pdf}
  \caption{Obrazki rastrowe i wektorowe}
\end{figure}

Rysunek 2.1 pochodzi z takiej tam strony\footnote{\url{http://siw.amu.edu.pl/siw/strona-glowna/strona-glowna}}.

\section{Znów podrozdział}

Przykład inline'owego trybu matematycznego: $X_{n+1} = a X_n + c \imod{m}$.

\subsection{Ostatni poziom zagłębienia uwzględniany w domyślnym spisie treści.}

% Cytat na wstęp sekcji.
\begin{chapquote}{Marcin Mateusz Hanc}
gracze na szczęście nie przejmowali się dotarciem do ``końca świata'' \cite{Hanc15szumy}
\end{chapquote}

% Przykład tabeli z danymi.
\begin{table}[h]
\centering
\label{tab:RNG_examples}
\begin{tabular}{l|c}
 Nagłówek & Coś innego \\
 \hline \hline
 Pierwszy rząd & 100 \\
 \hline
 Rząd & \multirow{2}{*}{TAK} \\
 podwójny & \\
\end{tabular}
\caption{Przykładowa tabela}
\end{table}

% Przykład wstawienia listingu kodu w języku C++ z podpisem.
\lstinputlisting[language=C++, captionpos=b, belowcaptionskip=4pt, caption={Ułamkowy Ruch Browna}]{fBm.cpp}

\section{Sekcja ze znakiem \_ działa}

% Wyliczenie elementów.
\begin{enumerate}
\item Elementy mogą być dłuższe niż jedna linia.

FAKTYCZNIE.

\item Drugi element.

\end{enumerate}

\chapter{Ostatni rozdział}

Jakieś treści podsumowujące pracę. Być może wnioski i pomysły na badania możliwe do przeprowadzenia na podstawie owej pracy.

\newpage
% Dodanie wpisu Bibliografia do Spisu Treści.
\addcontentsline{toc}{chapter}{Bibliografia}
% Styl bibliografii: unsrt lub plain
\bibliographystyle{plain}
\bibliography{bibliografia}

% W przypadku dodawania dodatków:
\appendix
\addcontentsline{toc}{chapter}{Dodatki}

% Pojedyńczy dodatek.
\addcontentsline{toc}{section}{Dodatek A: Cośtam dodatkowego}
\chapter*{Dodatek A: Cośtam dodatkowego}

Zawartość tego czegoś dodatkowego.

% Spis akronimów użytych w pracy
\chapter*{Akronimy}
\begin{acronym}
\acro{KISS}{Keep It Simple Stupid}
\acro{IT}{Information Technology}
\end{acronym}
\end{document}
